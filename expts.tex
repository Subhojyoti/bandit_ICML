\begin{figure}[!tbp]
    \centering
    \begin{tabular}{ccc}
    \subfigure[0.32\textwidth][Experiment $1$: $20$ Bernoulli-distributed arms with $r_{i_{{i}\neq {*}}}=0.07$ and $r^{*}=0.1$.]
    {
    		\pgfplotsset{
		tick label style={font=\Huge},
		label style={font=\Huge},
		legend style={font=\Large},
		}
        \begin{tikzpicture}[scale=0.4]
      	\begin{axis}[
		xlabel={timestep},
		ylabel={Cumulative Regret},
		grid=major,
        %clip mode=individual,grid,grid style={gray!30},
        clip=true,
        %clip mode=individual,grid,grid style={gray!30},
  		legend style={at={(0.5,1.2)},anchor=north, legend columns=3} ]
      	% UCB
		\addplot table{results/Expt1/UCB_Vcomp_subsampled.txt};
		\addplot table{results/Expt1/DMEDcomp_subsampled.txt};
		\addplot table{results/Expt1/KLUCBcomp_subsampled.txt};
		\addplot table{results/Expt1/MOSScomp_subsampled.txt};
		\addplot table{results/Expt1/UCB1comp_subsampled.txt};
		\addplot table{results/Expt1/TScomp_subsampled.txt};
		\addplot table{results/Expt1/EclUCB01comp_subsampled.txt};
      	\legend{UCB-V,DMED,KL-UCB,MOSS,UCB1,TS,EClusUCB(p=4)}      	
      	\end{axis}
      	\end{tikzpicture}
  		\label{fig:1}
    }
    &
    \subfigure[0.32\textwidth][Experiment $2$: $100$ Gaussian-distributed arms with $r_{i_{{i}\neq {*}:1-33}}=0.01$, $r_{i_{{i}\neq {*}:34-99}}=0.06$ and $r^{*}_{i=100}=0.1$. ]
    {
    		\pgfplotsset{
		tick label style={font=\Huge},
		label style={font=\Huge},
		legend style={font=\Large},
		}
        \begin{tikzpicture}[scale=0.4]
        \begin{axis}[
		xlabel={timestep},
		ylabel={Cumulative Regret},
        %clip mode=individual,grid,grid style={gray!30},
       	grid=major,
       	clip=true,
  		legend style={at={(0.5,1.2)},anchor=north, legend columns=3} ]
      	% UCB
        \addplot table{results/Expt2/UCB1comp_subsampled.txt};
		\addplot table{results/Expt2/clucb_p_10_comp_subsampled.txt};
		\addplot table{results/Expt2/Med_Elimcomp_subsampled.txt};
		\addplot table{results/Expt2/AclUCB01_comp_subsampled.txt};
		\addplot table{results/Expt2/MOSScomp_subsampled.txt};
		\addplot table{results/Expt2/EclUCB02_comp_subsampled.txt};
		\addplot table{results/Expt2/UCB_Improvedcomp_subsampled.txt};
      	\legend{UCB1,ClusUCB(p=10),Med-Elim,AClusUCB,MOSS,EClusUCB(p=10),UCB-Improved}
      	\end{axis}
      	\end{tikzpicture}
   		\label{fig:2}
    }
    &
    \subfigure[0.32\textwidth][Experiment $3$: Experiment $3$: $20$ to $90$ Bernoulli-distributed arms with $r_{i_{{i}\neq {*}}}=0.05$ and $r^{*}=0.1$. ]
    {
    	\pgfplotsset{
		tick label style={font=\Huge},
		label style={font=\Huge},
		legend style={font=\Large},
		}
        \begin{tikzpicture}[scale=0.4]
        \begin{axis}[
		xlabel={timestep},
		ylabel={Cumulative Regret},
        %clip mode=individual,grid,grid style={gray!30},
		grid=major,
		clip=true,
  		legend style={at={(0.5,1.2)},anchor=north, legend columns=3} ]
        % UCB
		\addplot table{results/Expt2/UCB1comp_subsampled.txt};
		\addplot table{results/Expt2/clucb_p_10_comp_subsampled.txt};
		\addplot table{results/Expt2/Med_Elimcomp_subsampled.txt};
		\addplot table{results/Expt2/AclUCB01_comp_subsampled.txt};
		\addplot table{results/Expt2/MOSScomp_subsampled.txt};
		\addplot table{results/Expt2/EclUCB02_comp_subsampled.txt};
		\addplot table{results/Expt2/UCB_Improvedcomp_subsampled.txt};
      	\legend{UCB1,ClusUCB(p=10),Med-Elim,AClusUCB,MOSS,EClusUCB(p=10),UCB-Improved}
      	\end{axis}
        \end{tikzpicture}
        \label{fig:3}
    }
    \end{tabular}
    %\caption{Cumulative regret for various bandit algorithms on three stochastic K-armed bandit environments. 
    \label{fig:karmed}
\end{figure}


%\begin{figure*}
%\centering
%  \begin{tabular}{ccc}
%%%%%% expt1  
%  \begin{subfigure}{0.32\textwidth}
% \tabl{c}{\scalebox{0.6}{\begin{tikzpicture}
%      \begin{axis}[
%	xlabel={timestep},
%	ylabel={Cumulative regret},
%       clip mode=individual,grid,grid style={gray!30},
%  legend style={at={(0.5,-0.2)},anchor=north,legend columns=3} ]  
%  % UCB    
%
%\addplot table[x index=0,y index=1,col sep=tab,each nth point={10}] {results/Expt1/UCB_Vcomp_subsampled.txt};
%\addplot table[x index=0,y index=1,col sep=tab,each nth point={10}] {results/Expt1/DMEDcomp_subsampled.txt};
%\addplot table[x index=0,y index=1,col sep=tab,each nth point={10}] {results/Expt1/KLUCBcomp_subsampled.txt};
%\addplot table[x index=0,y index=1,col sep=tab,each nth point={10}] {results/Expt1/MOSScomp_subsampled.txt};
%\addplot table[x index=0,y index=1,col sep=tab,each nth point={10}] {results/Expt1/UCB1comp_subsampled.txt};
%\addplot table[x index=0,y index=1,col sep=tab,each nth point={10}] {results/Expt1/TScomp_subsampled.txt};
%\addplot table[x index=0,y index=1,col sep=tab,each nth point={10}] {results/Expt1/EclUCB01comp_subsampled.txt};
%      \legend{UCB-V,DMED,KL-UCB,MOSS,UCB1,TS,EClusUCB(p=4)}
%      \end{axis}
%      \end{tikzpicture}}\\}
%			\caption{Experiment $1$: $20$ Bernoulli-distributed arms with $r_{i_{{i}\neq {*}}}=0.07$ and $r^{*}=0.1$.}
%  \label{fig:1}
%  \end{subfigure}
%	&
%	%%%%%%% Expt 2
%	  \begin{subfigure}{0.32\textwidth}
% \tabl{c}{\scalebox{0.6}{\begin{tikzpicture}
%      \begin{axis}[
%	xlabel={timestep},
%	ylabel={Cumulative regret},
%       clip mode=individual,grid,grid style={gray!30},
%  legend style={at={(0.5,-0.2)},anchor=north,legend columns=3} ]
%      % UCB
%\addplot table[x index=0,y index=1,col sep=tab,each nth point={10}] {results/Expt2/UCB1comp_subsampled.txt};
%\addplot table[x index=0,y index=1,col sep=tab,each nth point={10}] {results/Expt2/clucb_p_10_comp_subsampled.txt};
%\addplot table[x index=0,y index=1,col sep=tab,each nth point={10}] {results/Expt2/Med_Elimcomp_subsampled.txt};
%\addplot table[x index=0,y index=1,col sep=tab,each nth point={10}] {results/Expt2/AclUCB01_comp_subsampled.txt};
%\addplot table[x index=0,y index=1,col sep=tab,each nth point={10}] {results/Expt2/MOSScomp_subsampled.txt};
%\addplot table[x index=0,y index=1,col sep=tab,each nth point={10}] {results/Expt2/EclUCB02_comp_subsampled.txt};
%\addplot table[x index=0,y index=1,col sep=tab,each nth point={10}] {results/Expt2/UCB_Improvedcomp_subsampled.txt};
%      \legend{UCB1,ClusUCB(p=10),Med-Elim,AClusUCB,MOSS,EClusUCB(p=10),UCB-Improved}
%      \end{axis}
%      \end{tikzpicture}}\\}
%			\caption{Experiment $2$: $100$ Gaussian-distributed arms with $r_{i_{{i}\neq {*}:1-33}}=0.01$, $r_{i_{{i}\neq {*}:34-99}}=0.06$ and $r^{*}_{i=100}=0.1$.}
%  \label{fig:2}
%  \end{subfigure}
%	&
%	%%%%%%% Expt 3
%	  \begin{subfigure}{0.32\textwidth}
% \tabl{c}{\scalebox{0.6}{\begin{tikzpicture}
%      \begin{axis}[
%	xlabel={Arms},
%	ylabel={Cumulative regret},
%       clip mode=individual,grid,grid style={gray!30},
%  legend style={at={(0.5,-0.2)},anchor=north,legend columns=-1} ]
%      % UCB
%%\addplot table[x index=0,y index=1,col sep=tab] {results/Expt3/clUCB20_500.txt};
%\addplot table[x index=0,y index=1,col sep=tab] {results/Expt3_1/EclUCB20_90.txt};
%\addplot table[x index=0,y index=1,col sep=tab] {results/Expt3_1/MOSS20_90.txt};
%\addplot table[x index=0,y index=1,col sep=tab] {results/Expt3_1/OCUCB20_90.txt};
%      \legend{EClusUCB(p=K/10),MOSS,OCUCB}
%      \end{axis}
%      \end{tikzpicture}}\\}
%			\caption{Experiment $3$: $20$ to $90$ Bernoulli-distributed arms with $r_{i_{{i}\neq {*}}}=0.05$ and $r^{*}=0.1$.}
%  \label{fig:3}
%  \end{subfigure}
%  \end{tabular}
%\caption{Cumulative regret for various bandit algorithms on three stochastic K-armed bandit environments. 
%}
%\label{fig:karmed}
%\end{figure*}



%\begin{figure}[!tbp]
%\label{fig:1}
%\begin{minipage}[b]{0.5\textwidth}
%\includegraphics[width=\textwidth]{img/ClusUCB_variousAlgo.png}
%
%\caption{Experiment 1: Regret for various Algorithms. $T=60000$}
%\end{minipage}
%\end{figure}
%
%\hspace{0.1em}
%
%\begin{figure}[!tbp]
%\label{fig:2}
%\begin{minipage}[b]{0.5\textwidth}
%
%\includegraphics[width=\textwidth]{img/clusUCB_variousAlgo(expt2)_Final.png}
%\caption{Experiment 2: Regret for various Algorithms. $T=2\times 10^{6}$}
%\end{minipage}
%\end{figure}
%
%\hspace{0.1em}
%
%\begin{figure}[!tbp]
%\label{fig:3}
%\begin{minipage}[b]{0.5\textwidth}
%\includegraphics[width=\textwidth]{img/clUCB_MOSS_expt3.png}
%\caption{Experiment 3: Regret Growth for ClusUCB and MOSS . $T=10^{5} + K^{2}\times 10^{4}$ for $K=20$ to $200$}
%\end{minipage}
%\end{figure}
%
%\hspace{0.1em}
%


%In the stochastic bandit literature there are several powerful algorithms with and without proven regret bounds. Algorithms like $\epsilon$-greedy(\cite{sutton1998reinforcement}) or softmax(\cite{sutton1998reinforcement}) or UCB-Tuned(\cite{auer2002finite}) has no proven regret bounds. Again algorithms like UCB-$\delta$(\cite{abbasi2011improved}) with proven regret bound better than UCB1  falls within the realm of fixed confidence setting whereas one has to provide the probability of error $\delta$. We also make a distinction between frequentist based approach like the UCB algorithms and the Bayesian approach like the Thompson Sampling(\cite{agrawal2011analysis}). 
For the purpose of performance comparison using cumulative regret as the metric, we implement the following algorithms:  KL-UCB\cite{garivier2011kl}, DMED\cite{honda2010asymptotically}, MOSS\cite{audibert2009minimax}, UCB1\cite{auer2002finite}, UCB-Improved\cite{auer2010ucb}, Median Elimination\cite{even2006action}, Thompson Sampling(TS)\cite{agrawal2011analysis} and UCB-V\cite{audibert2009exploration}\footnote{The implementation for KL-UCB and DMED were taken from \cite{CapGarKau12}}. The parameters of ClusUCB algorithm for all the experiments are set as follows: $\psi=\log T$, $\rho_{s}=0.5$ and $\rho_{a}=0.25$. When $K$ is large and $p$ is small it is advantageous to run $\rho_{a} < \rho_{s}$(see 
Corollary \ref{Result:Corollary:2}) because this will aggressively eliminate arms within each cluster while full cluster elimination will be more conservative.
%So in all the experiments $\rho_{a}$ and $\rho_{s}$ are initialized to $1$ and then reduced after every round. By this definition of $\rho_{a},\rho_{s}$ we have made sure that their value always remain bounded $\in(0,1]$. 
% The intuition here is that since each cluster contains a large number of arms it should be eliminated less aggressively. 
%For parameter settings of ClusUCB and further experiments see Appendix \ref{App:Further:Expt}.

The first experiment is conducted over a testbed of $20$ arms for the test-cases involving Bernoulli reward distribution with expected rewards of the arms $r_{i_{{i}\neq {*}}}=0.07$ and $r^{*}=0.1$. These type of cases are frequently encountered in web-advertising domain. The horizon $T$ is set to $60000$. After limited exploratory experimentation the number of clusters $p$ for ClusUCB is set to $4$. The regret is averaged over $100$ independent runs and is shown in Figure \ref{fig:1}. EClusUCB, MOSS, UCB1, UCB-V, KL-UCB, TS and DMED are run in this experimental setup and we observe that EClusUCB performs better than all the aforementioned algorithms except TS. Because of the short horizon $T$, we do not implement UCB-Improved and Median Elimination on this test-case. We also observe that in this case the cumulative regret of EClusUCB and TS are almost similar to each other.

	The second experiment is conducted over a testbed of $100$ arms involving Gaussian reward distribution with expected rewards of the arms $r_{i_{{i}\neq {*}:1-33}}=0.01$, $r_{i_{{i}\neq {*}:34-99}}=0.06$ and $r^{*}_{i=100}=0.1$ with variance set at $\sigma^{2} = 0.3,\forall i\in A$. The horizon $T$ is set for a large duration of $2\times 10^{6}$ and the number of clusters $p=20$. The regret is averaged over $100$ independent runs and is shown in Figure \ref{fig:2}. In this case, in addition to EClusUCB, we also show the performance of ClusUCB algorithm (with $p=10$). From the results in Figure \ref{fig:2}, we observe that EClusUCB with $p=10$ outperforms ClusUCB with $p=10$ as well as MOSS, UCB1, UCB-Improved and Median-Elimination($\epsilon=0.03,\delta=0.1$). But as shown in Theorem \ref{Result:Theorem:1}, ClusUCB is better than UCB-Improved even though both are round based methods. Also the performance of UCB-Improved is poor in comparison to other algorithms, which is probably because of pulls wasted in initial exploration whereas ClusUCB with the choice of $\psi, \rho_{a}$ and $\rho_{s}$ performs much better.

	The third experiment is conducted over a testbed of $20-90$ (interval of $10$) arms with Bernoulli reward distribution, where the expected rewards of the arms are $r_{i_{{i}\neq {*}}}=0.05$ and $r^{*}=0.1$. The horizon $T$ is set to $10^{5} + K^{3}$ and the number of arms are increased from $K=20$ to $90$. The proposed algorithm EClusUCB is run with $p=K/10$. The regret is averaged over $100$ independent runs and is shown in Figure \ref{fig:3}. We report the performance of MOSS,OCUCB and EClusUCB only over this setup. From the results in Figure \ref{fig:3}, it is evident that the growth of regret for EClusUCB is lower than that of MOSS and OCUCB. 
%\begin{figure}
%\centering
%  \begin{tabular}{c}
%  %&
%  %%%%%%Expt4
%  \begin{subfigure}{0.45\textwidth}
% \tabl{c}{\scalebox{0.7}{\begin{tikzpicture}
%      \begin{axis}[
%	xlabel={timestep},
%	ylabel={Cumulative regret},
%       clip mode=individual,grid,grid style={gray!30},
%  legend style={at={(0.5,-0.2)},anchor=north, legend columns=4} ]
%      % UCB
%\addplot table[x index=0,y index=1,col sep=tab,each nth point={10}] {results/Expt4/clUCB1comp_subsampled.txt};
%\addplot table[x index=0,y index=1,col sep=tab,each nth point={10}] {results/Expt4/clUCB2comp_subsampled.txt};
%\addplot table[x index=0,y index=1,col sep=tab,each nth point={10}] {results/Expt4/clUCB3comp_subsampled.txt};
%\addplot table[x index=0,y index=1,col sep=tab,each nth point={10}] {results/Expt4/MOSScomp_subsampled.txt};
%\addplot table[x index=0,y index=1,col sep=tab,each nth point={10}] {results/Expt4/clUCB4comp_subsampled.txt};
%\addplot table[x index=0,y index=1,col sep=tab,each nth point={10}] {results/Expt4/clUCB5comp_subsampled.txt};
%\addplot table[x index=0,y index=1,col sep=tab,each nth point={10}] {results/Expt4/TScomp_subsampled.txt};
%      \legend{ClusUCB(1A),ClusUCB(4B),ClusUCB(10B),MOSS,ClusUCB(5S),ClusUCB(10S),TS}
%      %\legend{ClusUCB (NC, p=1),ClusUCB (C, p=4),ClusUCB(C, p=10) ,MOSS, ClusUCB(C, p=5, NAE), ClusUCB(C, p=10, NAE)}
%      %\legend{ClusUCB(1,A),ClusUCB(4,B),ClusUCB(10,B), MOSS,ClusUCB(5,S), ClusUCB(10,A)}
%      \end{axis}
%      \end{tikzpicture}}\\}
%			\caption{Experiment $4$: Cumulative regret for ClusUCB variants: $1,4,5,10$ correspond to number $p$ of clusters and A, S, B correspond to having only arm elimination, only cluster elimination and having both in Algorithm \ref{alg:clusucb}. For using just cluster elimination in ClusUCB, we stop when we left with only one cluster and play the max payoff arm of that cluster for the remaining time horizon. }
%  \label{Fig:variousClus}
%  \end{subfigure}
%  \end{tabular}
%\end{figure}


\begin{figure}
    \centering
    \begin{tabular}{c}
    \subfigure[0.45\textwidth][Experiment $4$: Cumulative regret for ClusUCB variants: $1,3,5,10,15,25$ correspond to number $p$ of clusters]
    {
    		\pgfplotsset{
		tick label style={font=\Huge},
		label style={font=\Huge},
		legend style={font=\Large},
		}
        \begin{tikzpicture}[scale=0.4]
      	\begin{axis}[
		xlabel={timestep},
		ylabel={Cumulative Regret},
		grid=major,
        %clip mode=individual,grid,grid style={gray!30},
        clip=true,
        %clip mode=individual,grid,grid style={gray!30},
  		legend style={at={(0.5,1.2)},anchor=north, legend columns=3} ]
      	% UCB
		\addplot table{results/Expt4_1/clucb1_comp_subsampled.txt};
		\addplot table{results/Expt4_1/clucb3_comp_subsampled.txt};
		\addplot table{results/Expt4_1/clucb5_comp_subsampled.txt};
		\addplot table{results/Expt4_1/clucb10_comp_subsampled.txt};
		\addplot table{results/Expt4_1/clucb15_comp_subsampled.txt};
		\addplot table{results/Expt4_1/clucb25_comp_subsampled.txt};
      	\legend{ClusUCB(1),ClusUCB(3),ClusUCB(5),ClusUCB(10),ClusUCB(15),ClusUCB(25)} 
      	\end{axis}
      	\end{tikzpicture}
  		\label{Fig:variousClus}
    }
	\end{tabular}
    %\caption{Experiment $4$: Cumulative regret for ClusUCB variants: $1,3,5,10,15,25$ correspond to number $p$ of clusters}
\end{figure}

%\begin{figure}
%\centering
%  \begin{tabular}{c}
%  %&
%  %%%%%%Expt4
%  \begin{subfigure}{0.45\textwidth}
% \tabl{c}{\scalebox{0.7}{\begin{tikzpicture}
%      \begin{axis}[
%	xlabel={timestep},
%	ylabel={Cumulative regret},
%       clip mode=individual,grid,grid style={gray!30},
%  legend style={at={(0.5,-0.2)},anchor=north, legend columns=4} ]
%      % UCB
%\addplot table[x index=0,y index=1,col sep=tab,each nth point={10}] {results/Expt4_1/clucb1_comp_subsampled.txt};
%\addplot table[x index=0,y index=1,col sep=tab,each nth point={10}] {results/Expt4_1/clucb3_comp_subsampled.txt};
%\addplot table[x index=0,y index=1,col sep=tab,each nth point={10}] {results/Expt4_1/clucb5_comp_subsampled.txt};
%\addplot table[x index=0,y index=1,col sep=tab,each nth point={10}] {results/Expt4_1/clucb10_comp_subsampled.txt};
%\addplot table[x index=0,y index=1,col sep=tab,each nth point={10}] {results/Expt4_1/clucb15_comp_subsampled.txt};
%\addplot table[x index=0,y index=1,col sep=tab,each nth point={10}] {results/Expt4_1/clucb25_comp_subsampled.txt};
%      \legend{ClusUCB(1),ClusUCB(3),ClusUCB(5),ClusUCB(10),ClusUCB(15),ClusUCB(25)} 
%      \end{axis}
%      \end{tikzpicture}}\\}
%			\caption{Experiment $4$: Cumulative regret for ClusUCB variants: $1,3,5,10,15,25$ correspond to number $p$ of clusters}
%%and A, S, B correspond to having only arm elimination, only cluster elimination and having both in Algorithm \ref{alg:clusucb}. For using just cluster elimination in ClusUCB, we stop when we left with only one cluster and play the max payoff arm of that cluster for the remaining time horizon. 
%  \label{Fig:variousClus}
%  \end{subfigure}
%  \end{tabular}
%\end{figure}

	The fourth experiment is performed over a testbed having $50$ Gaussian-distributed arms with $r_{i_{:{{i}\neq {*}}}}=0.8,\forall i\in A$, $r^{*}=0.9$ and $\sigma^{2}=1.0$. In Figure \ref{Fig:variousClus}, we report the results with $T=400000$ averaged over $100$ independent runs for ClusUCB with  $p=\lbrace 1,3,5,10,15,25\rbrace$. Also, in this experiment we take $\psi = K^{2}T$, $\rho_a=0.25$ and $\rho_{s}=0.5$ as stated in Corollary \ref{Result:Corollary:2}. The high variance leads to a greater number of errors committed by ClusUCB-AE that is ClusUCB($p=1$) but as proved in Proposition \ref{proofTheorem:Prop:1} the cumulative regret is lesser than  ClusUCB. But because of the increased errors committed in predicting the optimal arm and because of the large horizon, we eventually see that ClusUCB(p=$5,10,25,25$) outperforms ClusUCB-AE while ClusUCB($p=3$) regret is worse than ClusUCB-AE. The error percentage in the $6$ cases (in the order as shown in legend of Fig \ref{Fig:variousClus}) are $14,12,5,3,3$ and $3$. The range of p is shown to be between $\sqrt{\log K}$ to $\frac{K}{2}$ and as we approach $\frac{K}{2}$ we see that the error percentage stabilizes to $3\%$.
	More experiments are shown in Appendix \ref{App:MoreExp}.
	
	
%	Two other benchmark algorithms are considered here, MOSS which is one of our main competitors and TS which has performed near equivalently in experiment $1$(see Fig \ref{fig:1}). In this case we see that, since $\rho_{a}$ is decreased very fast, the optimal arm $a^{*}$ gets eliminated most of the time for no clustering $p=1$. While a balance of $p,\rho_{a}$ and $\rho_{s}$ gives a much better result. ClusUCB with $p=4$ and $10$ 
%perform better than MOSS, while $p=1$ with just arm elimination does not converge and $p=5,10$ with just Cluster elimination and no arm elimination also does not converge, implying both cluster and arm elimination are necessary. This also concurs with the theoretical observations (see Proposition \ref{proofTheorem:Prop:2} and the discussion in Section \ref{sec:results} on the analysis of elimination error).
%Note also that ClusUCB($p=4$) outperforms TS in this setup.	
	
%We also see that in this testbed UCB-Improved performs the worst and it confirms our assumption that it spends too much pulls in the initial exploration.

%We set $\psi=1$, $\rho_{s}=\frac{1}{2^{m+1}}$ and $\rho_{a}=\frac{1}{2^{2m+1}}$.

%The jumps in the graph for ClusUCB happens because of the error(eliminating optimal arm) and the margin of error(in red) is also shown in the graph. 
%\todos{I did not see this jump in the txt files shared for Expt 3}
