\begin{figure}[!tbp]
    \centering
    \begin{tabular}{cc}
    \setlength{\tabcolsep}{0.1pt}
    \subfigure[0.25\textwidth][Experiment $1$: $20$ Bernoulli-distributed arms with $r_{i_{{i}\neq {*}}}=0.07$ and $r^{*}=0.1$.]
    {
    		\pgfplotsset{
		tick label style={font=\Huge},
		label style={font=\Huge},
		legend style={font=\Large},
		ylabel style={yshift=32pt},
		%legend style={legendshift=32pt},
		}
        \begin{tikzpicture}[scale=0.35]
      	\begin{axis}[
		xlabel={timestep},
		ylabel={Cumulative Regret},
		grid=major,
        %clip mode=individual,grid,grid style={gray!30},
        clip=true,
        %clip mode=individual,grid,grid style={gray!30},
  		legend style={at={(0.5,1.5)},anchor=north, legend columns=3} ]
      	% UCB
		\addplot table{results/NewExpt/Expt1/UCBV01_comp_subsampled.txt};
		\addplot table{results/NewExpt/Expt1/DMED01_comp_subsampled.txt};
		\addplot table{results/NewExpt/Expt1/KLUCB01_comp_subsampled.txt};
		\addplot table{results/NewExpt/Expt1/MOSS01_comp_subsampled.txt};
		\addplot table{results/NewExpt/Expt1/UCB01_comp_subsampled.txt};
		\addplot table{results/NewExpt/Expt1/OCUCB01_comp_subsampled.txt};
		\addplot table{results/NewExpt/Expt1/TS01_comp_subsampled.txt};
		\addplot table{results/NewExpt/Expt1/EclUCB01_comp_subsampled.txt};
		\addplot table{results/NewExpt/Expt1/EclUCB011_comp_subsampled.txt};
      	\legend{UCB-V,DMED,KL-UCB,MOSS,UCB1,OCUCB,TS,EClusUCBA,EClusUCB}      	
      	\end{axis}
      	\end{tikzpicture}
  		\label{fig:1}
    }
    &
    \subfigure[0.25\textwidth][Experiment $2$: $100$ Gaussian-distributed arms with $r_{i_{{i}\neq {*}:1-33}}=0.01$, $r_{i_{{i}\neq {*}:34-99}}=0.06$ and $r^{*}_{i=100}=0.1$. ]
    {
    		\pgfplotsset{
		tick label style={font=\Huge},
		label style={font=\Huge},
		legend style={font=\Large},
		}
        \begin{tikzpicture}[scale=0.35]
        \begin{axis}[
		xlabel={timestep},
		ylabel={Cumulative Regret},
        %clip mode=individual,grid,grid style={gray!30},
       	grid=major,
       	clip=true,
  		legend style={at={(0.5,1.5)},anchor=north, legend columns=3} ]
      	% UCB
        \addplot table{results/NewExpt/Expt2_1/UCB01_comp_subsampled.txt};
		\addplot table{results/NewExpt/Expt2_1/clUCB01_comp_subsampled.txt};
		\addplot table{results/NewExpt/Expt2_1/MedElim01_comp_subsampled.txt};
		\addplot table{results/NewExpt/Expt2_1/MOSS01_comp_subsampled.txt};
		\addplot table{results/NewExpt/Expt2_1/EclUCB01_comp_subsampled.txt};
		\addplot table{results/NewExpt/Expt2_1/UCBR01_comp_subsampled.txt};
		\addplot table{results/NewExpt/Expt2_1/OCUCB01_comp_subsampled.txt};
		\addplot table{results/NewExpt/Expt2_1/EclUCB01_p_1_comp_subsampled.txt};
      	\legend{UCB1,ClusUCBA,Med-Elim,MOSS,EClusUCBA,UCB-Imp,OCUCB,EClusUCBA-AE}
      	\end{axis}
      	\end{tikzpicture}
   		\label{fig:2}
    }
    \end{tabular}
    \caption{Cumulative regret for various bandit algorithms on two stochastic K-armed bandit environments. }
    \label{fig:karmed}
\end{figure}

For the purpose of performance comparison using cumulative regret as the metric, we implement the following algorithms:  KL-UCB\cite{garivier2011kl}, DMED\cite{honda2010asymptotically}, MOSS\cite{audibert2009minimax}, UCB1\cite{auer2002finite}, UCB-Improved\cite{auer2010ucb}, Median Elimination\cite{even2006action}, Thompson Sampling(TS)\cite{agrawal2011analysis} and UCB-V\cite{audibert2009exploration}\footnote{The implementation for KL-UCB and DMED were taken from \cite{CapGarKau12}}. The parameters of EClusUCB algorithm for all the experiments are set as follows: $\psi=\frac{T}{\log K}$, $\rho_{s}=0.5$, $\rho_{a}=0.5$ and $p=\lceil\frac{K}{\log K}\rceil$ (as in Corollary \ref{Result:Corollary:2}) whereas for ClusUCB(Aggressive)$\big/$EClusUCB(Aggressive) or ClusUCBA$\big/$EClusUCBA algorithm the parameters are set as $\psi=\log T$, $\rho_{s}=0.5$, $\rho_{a}=0.5$ and $p=\lceil\frac{K}{\log K}\rceil$. We can clearly see that EClusUCBA has a lower exploration regulatory factor and conducts less exploration and is a riskier algorithm than EClusUCB.
%When $K$ is large an2d $p$ is small, it is advantageous to run $\rho_{a} < \rho_{s}$(see 
%Corollary \ref{Result:Corollary:}) because this will aggressively eliminate arms within each cluster while full cluster elimination will be more conservative.
%So in all the experiments $\rho_{a}$ and $\rho_{s}$ are initialized to $1$ and then reduced after every round. By this definition of $\rho_{a},\rho_{s}$ we have made sure that their value always remain bounded $\in(0,1]$. 
% The intuition here is that since each cluster contains a large number of arms it should be eliminated less aggressively. 
%For parameter settings of ClusUCB and further experiments see Appendix \ref{App:Further:Expt}.

The first experiment is conducted over a testbed of $20$ arms for the test-cases involving Bernoulli reward distribution with expected rewards of the arms $r_{i_{{i}\neq {*}}}=0.07$ and $r^{*}=0.1$. These type of cases are frequently encountered in web-advertising domain. The horizon $T$ is set to $60000$. 
%After limited exploratory experimentation the number of clusters $p$ for ClusUCB is set to $4$. 
The regret is averaged over $100$ independent runs and is shown in Figure \ref{fig:1}. EClusUCB,EClusUCBA, MOSS, UCB1, UCB-V, KL-UCB, TS and DMED are run in this experimental setup and we observe that EClusUCBA performs better than all the aforementioned algorithms. Because of the short horizon $T$, we do not implement UCB-Improved and Median Elimination on this test-case. 
%We also observe that in this case the cumulative regret of EClusUCB and TS are almost similar to each other.

	The second experiment is conducted over a testbed of $100$ arms involving Gaussian reward distribution with expected rewards of the arms $r_{i_{{i}\neq {*}:1-33}}=0.01$, $r_{i_{{i}\neq {*}:34-99}}=0.06$ and $r^{*}_{i=100}=0.1$ with variance set at $\sigma_{i}^{2} = 0.3,\forall i\in A$. The horizon $T$ is set for a large duration of $2\times 10^{6}$ and the regret is averaged over $100$ independent runs and is shown in Figure \ref{fig:2}. In this case, in addition to EClusUCBA, we also show the performance of ClusUCBA algorithm. From the results in Figure \ref{fig:2}, we observe that EClusUCBA outperforms ClusUCBA as well as MOSS, UCB1, UCB-Improved and Median-Elimination($\epsilon=0.03,\delta=0.1$). But as shown in Theorem \ref{Result:Theorem:1}, ClusUCBA is better than UCB-Improved even though both are round based methods. Also the performance of UCB-Improved is poor in comparison to other algorithms, which is probably because of pulls wasted in initial exploration whereas ClusUCBA with the choice of $\psi, \rho_{a}$ and $\rho_{s}$ performs much better. More experiments are shown in Appendix \ref{App:MoreExp}.

%	The third experiment is conducted over a testbed of $20-90$ (interval of $10$) arms with Bernoulli reward distribution, where the expected rewards of the arms are $r_{i_{{i}\neq {*}}}=0.05$ and $r^{*}=0.1$. The horizon $T$ is set to $10^{5} + K^{3}$ and the number of arms are increased from $K=20$ to $90$. The proposed algorithm EClusUCB is run with $p=K/10$. The regret is averaged over $100$ independent runs and is shown in Figure \ref{fig:3}. We report the performance of MOSS,OCUCB and EClusUCB only over this setup. From the results in Figure \ref{fig:3}, it is evident that the growth of regret for EClusUCB is lower than that of MOSS and OCUCB. 
%
%\begin{figure}
%    \centering
%    \begin{tabular}{cc}
%    \subfigure[0.25\textwidth][Experiment $3$: Experiment $3$: $20$ to $90$ Bernoulli-distributed arms with $r_{i_{{i}\neq {*}}}=0.05$ and $r^{*}=0.1$. ]
%    {
%    	\pgfplotsset{
%		tick label style={font=\Huge},
%		label style={font=\Huge},
%		legend style={font=\Large},
%		}
%        \begin{tikzpicture}[scale=0.3]
%        \begin{axis}[
%		xlabel={timestep},
%		ylabel={Cumulative Regret},
%        %clip mode=individual,grid,grid style={gray!30},
%		grid=major,
%		clip=true,
%  		legend style={at={(0.5,-0.3)},anchor=north, legend columns=3} ]
%        % UCB
%		\addplot table{results/Expt3_1/EclUCB20_90.txt};
%		\addplot table{results/Expt3_1/MOSS20_90.txt};
%		\addplot table{results/Expt3_1/OCUCB20_90.txt};
%      	\legend{UCB1,ClusUCB,Med-Elim,AClusUCB,MOSS,EClusUCB,UCB-Improved}
%      	\end{axis}
%        \end{tikzpicture}
%        \label{fig:3}
%    }
%    &
%    \subfigure[0.25\textwidth][Experiment $4$: Cumulative regret for ClusUCB variants: $1,3,5,10,15,25$ correspond to number $p$ of clusters]
%    {
%    		\pgfplotsset{
%		tick label style={font=\Huge},
%		label style={font=\Huge},
%		legend style={font=\Large},
%		}
%        \begin{tikzpicture}[scale=0.3]
%      	\begin{axis}[
%		xlabel={timestep},
%		ylabel={Cumulative Regret},
%		grid=major,
%        %clip mode=individual,grid,grid style={gray!30},
%        clip=true,
%        %clip mode=individual,grid,grid style={gray!30},
%  		legend style={at={(0.5,-0.3)},anchor=north, legend columns=3} ]
%      	% UCB
%		\addplot table{results/Expt4_1/clucb1_comp_subsampled.txt};
%		\addplot table{results/Expt4_1/clucb3_comp_subsampled.txt};
%		\addplot table{results/Expt4_1/clucb5_comp_subsampled.txt};
%		\addplot table{results/Expt4_1/clucb10_comp_subsampled.txt};
%		\addplot table{results/Expt4_1/clucb15_comp_subsampled.txt};
%		\addplot table{results/Expt4_1/clucb25_comp_subsampled.txt};
%      	\legend{ClusUCB(1),ClusUCB(3),ClusUCB(5),ClusUCB(10),ClusUCB(15),ClusUCB(25)} 
%      	\end{axis}
%      	\end{tikzpicture}
%  		\label{Fig:variousClus}
%    }
%	\end{tabular}
%    %\caption{Experiment $4$: Cumulative regret for ClusUCB variants: $1,3,5,10,15,25$ correspond to number $p$ of clusters}
%\end{figure}
%
%
%
%	The fourth experiment is performed over a testbed having $50$ Gaussian-distributed arms with $r_{i_{:{{i}\neq {*}}}}=0.8,\forall i\in A$, $r^{*}=0.9$ and $\sigma^{2}=1.0$. In Figure \ref{Fig:variousClus}, we report the results with $T=400000$ averaged over $100$ independent runs for ClusUCB with  $p=\lbrace 1,3,5,10,15,25\rbrace$. Also, in this experiment we take $\psi = K^{2}T$, $\rho_a=0.25$ and $\rho_{s}=0.5$ as stated in Corollary \ref{Result:Corollary:2}. The high variance leads to a greater number of errors committed by ClusUCB-AE that is ClusUCB($p=1$) but as proved in Proposition \ref{proofTheorem:Prop:1} the cumulative regret is lesser than  ClusUCB. But because of the increased errors committed in predicting the optimal arm and because of the large horizon, we eventually see that ClusUCB(p=$5,10,25,25$) outperforms ClusUCB-AE while ClusUCB($p=3$) regret is worse than ClusUCB-AE. The error percentage in the $6$ cases (in the order as shown in legend of Fig \ref{Fig:variousClus}) are $14,12,5,3,3$ and $3$. The range of p is shown to be between $\sqrt{\log K}$ to $\frac{K}{2}$ and as we approach $\frac{K}{2}$ we see that the error percentage stabilizes to $3\%$.
	
	
	
